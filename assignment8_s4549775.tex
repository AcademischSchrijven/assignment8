\documentclass[12pt, a4paper]{article}

\usepackage[backend=biber]{biblatex}
\addbibresource{references.bib}

\setlength\parskip{1em}
\setlength\parindent{0em}

\title{Assignment 8}

\author{Hendrik Werner s4549775}

\begin{document}
\maketitle

\section{Abstract}
In this paper I present an easy to understand and follow approach to detect an object's position, and the direction it is facing, from visual input using OpenCV. This is done as a case study of an autonomous football playing robot.

\section{Introduction}
For a university project I wanted to detect the position of a robot and which way it was facing, as well as the position of a ball, using visual input. I looked into computer vision frameworks and eventually chose OpenCV. I expected there to be some prebuilt functionality for something as mundane as detecting an arrow but was disappointed.

In this article I want to present the approach I took to detect the features I needed; and how this was applied in a robotics project. This article is positioned as an introduction into image detection and does not assume familiarity with the subject. Concepts and terminology are introduced as needed; it serves as a quick overview over image detection and OpenCV in particular.

\section{Problem}
As input for my robot's decision making algorithm I need the position of the ball, as well as the position of the robot, and the direction it faces. The ball is round, so direction does not really apply to it.

For this purpose I mounted a camera above the playing field facing straight down. The ball is a bright yellow, and on the robot I mounted a bright red arrow, pointing forward.

So the problem we need to solve is the following: Given visual input with unique, and brightly colored objects, how can the positions and directions of the objects be extracted?

\section{Overview over Computer Vision}
Computer Vision is the field of study that deals with the extraction of high level features from digital visual data. In our case this data is the video stream from the camera, and the features are the positions and directions of the ball and car.

There are many applications for Computer Vision, most of which are concerned with automation of tasks previously done by humans. In our case playing football, though there are of course more practical applications as well.
Tesla heavily invests in Computer Vision for its application in self driving vehicle technology \cite{teslaAutopilot} \cite{teslaCvArticle}.

\section{Overview over OpenCV}
OpenCV (Open Computer Vision) is a cross-platform, open source computer vision library. Development was started by Intel in 1999 as a research project. It is released under the BSD license which allows for commercial and non-commercial free use \cite{learningOpenCV}.

Today it is one of the industry standards for Computer Vision, used and sponsored by companies such as Google, Microsoft, and Intel, among many others. It has interfaces for C, C++, Python, Java, and MATLAB \cite{aboutOpenCV}.

OpenCV is highly optimized and geared towards real time systems \cite{aboutOpenCV}. This is important for many applications in automation. Your car must be able to process visual information in real time to avoid collisions, for example.
For out purposes this is not of great importance but it is nice to be able to visualize the algorithms working. Quicker response times are also generally nice.

\section{Solution}

\section{Application}

\section{Conclusion}

\section{Further Reading}
\begin{itemize}
	\item The OpenCV 3 Cookbook \cite{openCVCookbook} is a complete introduction into the third version of the OpenCV library. "You will be presented with a variety of computer vision algorithms and exposed to important concepts in image and video analysis that will enable you to build your own computer vision applications." \cite{openCVCookbookWebsite}

	\item If you have Python experience you might want to look at OpenCV Computer Vision with Python \cite{openCVPython}, which focuses on Python.

	"This book has practical, project-based tutorials for Python developers and hobbyists who want to get started with computer vision with OpenCV and Python. It is a hands-on guide that covers the fundamental tasks of computer vision, capturing, filtering, and analyzing images, with step-by-step instructions for writing both an application and reusable library classes." \cite{openCVPythonWebsite}
\end{itemize}

\printbibliography

\end{document}
