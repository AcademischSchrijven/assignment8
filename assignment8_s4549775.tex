\documentclass[12pt, a4paper]{article}

\usepackage[backend=biber]{biblatex}
\addbibresource{references.bib}
\usepackage{listings}
\lstset{
	language=Python,
	breaklines=true,
}

\setlength\parskip{1em}
\setlength\parindent{0em}

\title{Assignment 8}

\author{Hendrik Werner s4549775}

\begin{document}
\maketitle

\section{Abstract}
In this paper I present an easy to understand and follow approach to detect an object's position, and the direction it is facing, from visual input using OpenCV. This is done as a case study of an autonomous football playing robot.

\section{Introduction}
For a university project I wanted to detect the position of a robot and which way it was facing, as well as the position of a ball, using visual input. I looked into computer vision frameworks and eventually chose OpenCV. I expected there to be some prebuilt functionality for something as mundane as detecting an arrow but was disappointed.

In this article I want to present the approach I took to detect the features I needed; and how this was applied in a robotics project. This article is positioned as an introduction into image detection and does not assume familiarity with the subject. Concepts and terminology are introduced as needed; it serves as a quick overview over image detection and OpenCV in particular.

\section{Problem}
As input for my robot's decision making algorithm I need the position of the ball, as well as the position of the robot, and the direction it faces. The ball is round, so direction does not really apply to it.

For this purpose I mounted a camera above the playing field facing straight down. The ball is a bright yellow, and on the robot I mounted a bright red arrow, pointing forward.

So the problem we need to solve is the following: Given visual input with unique, and brightly colored objects, how can the positions and directions of the objects be extracted?

\section{Overview over Computer Vision}
Computer Vision is the field of study that deals with the extraction of high level features from digital visual data. In our case this data is the video stream from the camera, and the features are the positions and directions of the ball and car.

There are many applications for Computer Vision, most of which are concerned with automation of tasks previously done by humans. In our case playing football, though there are of course more practical applications as well.
Tesla heavily invests in Computer Vision for its application in self driving vehicle technology \cite{teslaAutopilot} \cite{teslaCVArticle}.

\section{Overview over OpenCV}
OpenCV (Open Computer Vision) is a cross-platform, open source computer vision library. Development was started by Intel in 1999 as a research project. It is released under the BSD license which allows for commercial and non-commercial free use \cite{learningOpenCV}.

Today it is one of the industry standards for Computer Vision, used and sponsored by companies such as Google, Microsoft, and Intel, among many others. It has interfaces for C, C++, Python, Java, and MATLAB \cite{aboutOpenCV}.

OpenCV is highly optimized and geared towards real time systems \cite{aboutOpenCV}. This is important for many applications in automation. Your car must be able to process visual information in real time to avoid collisions, for example.
For out purposes this is not of great importance but it is nice to be able to visualize the algorithms working. Quicker response times are also generally nice.

The OpenCV API provides facilities for image (pre)processing, persistence, clustering, hight-level GUI, video analysis, camera calibration, 2d-feature extraction, object detection, Machine Learning, GPU acceleration, computational photography, image stitching, OpenCL acceleration, super resolution, 3d-visualization and even some non-free components \cite{openCVNonFree}.

In addition to that there are some deprecated modules and experimental / contributed functionality. \cite{openCVRefMan}

\section{Solution}
\subsection{Preprocessing}
OpenCV reads the image in as BGR (blue, green, red), where a color's hue is split across 3 different channels. We want to filter on the hue, so it is the easiest solution to convert the color to HSV (hue, saturation, value), which packs hoe information into a single channel.

\begin{lstlisting}
image_hsv = cv2.cvtColor(image, cv2.COLOR_BGR2HSV)
\end{lstlisting}

The images we capture from the camera are pretty noisy, so we apply a median blur. A bilateral filter is even better because it reduces noise while retaining edges. For out purposes a median blur is sufficient, so we use it because it's performance is better. \cite{openCVImageSmoothing}

\begin{lstlisting}
image_hsv = cv2.medianBlur(image_hsv, 3)
\end{lstlisting}

I made a generic object finder class, that does the following things:
\begin{enumerate}
	\item Make a mask. This is a black and white image of the same dimensions as the input image, in which all pixels falling within a certain color range are white, and the rest is black. It basically maps all pixels to whether or not they are in the color range.

	We use OpenCV's $inRange$ function for this:

	\begin{lstlisting}
mask = cv2.inRange(
	image_hsv,
	threshold["lower"],
	threshold["upper"],
)
	\end{lstlisting}
	\item Find the contours. The contours are the outlines of all the shapes in our mask.

	We use $findContours$:

	\begin{lstlisting}
contours = cv2.findContours(
	mask,
	cv2.RETR_LIST,
	cv2.CHAIN_APPROX_SIMPLE,
)
	\end{lstlisting}
	\item Filter out all contours with an area below some threshold:

	\begin{lstlisting}
contours = [
	c for c in contours
	if cv2.contourArea(c) >= minimum_area
]
	\end{lstlisting}
\end{enumerate}

\subsection{Finding the ball's position}

\subsection{Finding the robot's position and direction}

\section{Application}
We use the information we extracted as input for an algorithm that decides on a plan of action, that describes what our robot is supposed to do. This information is sent to an executor, which controls the robot via Bluetooth.

Even though the techniques were basic, and we extracted very limited information from the camera input, this was enough to make an arbitrarily placed robot find an arbitrarily placed ball, as long as they are visible to the camera.

This demonstrates the ease with which Computer Vision can be deployed even by amateurs, with the help of frameworks such as OpenCV.

\section{Conclusion}
OpenCV is a great tool for Computer Vision, which provides a basis to build sophisticated

The documentation isn't very comprehensive, especially for non-native interfaces of OpenCV. This can make in unnecessarily hard for beginners to get into the matter. With the right reading material however, it can be easy to quickly achieve great results, even for beginners; by building on the solid foundation of the OpenCV API.

The application for Computer Vision are diverse and it can be used from little hobby or school projects, to building the next generation of self driving vehicles.

\section{Further Reading}
\begin{itemize}
	\item The full source code, the planner, the executor, the specifications for building the robot, and additional documentation can be found on the project's GitHub page\cite{projectGithub}.
	\item The OpenCV 3 Cookbook \cite{openCVCookbook} is a complete introduction into the third version of the OpenCV library. "You will be presented with a variety of computer vision algorithms and exposed to important concepts in image and video analysis that will enable you to build your own computer vision applications." \cite{openCVCookbookWebsite}

	\item If you have Python experience you might want to look at OpenCV Computer Vision with Python \cite{openCVPython}, which focuses on Python.

	"This book has practical, project-based tutorials for Python developers and hobbyists who want to get started with computer vision with OpenCV and Python. It is a hands-on guide that covers the fundamental tasks of computer vision, capturing, filtering, and analyzing images, with step-by-step instructions for writing both an application and reusable library classes." \cite{openCVPythonWebsite}

	\item If you are more of a C++ fan, or you want to use OpenCV through its native interface, maybe you are interested in Learning OpenCV 3 Application Development \cite{openCVApplicationDevelopment}.

	"This book provides the steps to build and deploy an end-to-end application in the domain of computer vision using OpenCV/C++.". Topics covered are: how images are stored and processed by OpenCV, OpenCV-specific jargon, image traversal and pixel-wise operations, filtering, thresholding, edge detection, face detection, and much more. \cite{openCVApplicationDevelopmentWebsite}
\end{itemize}

Keep in mind that the C++ interface has much better documentation because of being the native language of OpenCV. However, generally speaking, Python is used for a lot of automation and Machine Learning, and provides a wealth of other supporting libraries for a wide variety of tasks. Python is also considered to be more concise than C++, which is payed for in reduced performance.

Do not focus too much on the choice of languages though, not even the choice of frameworks. The general concepts, algorithms, and approaches are much more generic. They will serve you well no matter the language or framework you happen to be working in.

\printbibliography

\end{document}
