\documentclass[12pt, a4paper]{article}

\setlength\parskip{1em}
\setlength\parindent{0em}

\title{Assignment 8}

\author{Hendrik Werner s4549775}

\begin{document}
\maketitle

\section{Abstract}

\section{Introduction}
For a university project I wanted to detect the position of a robot and which way it was facing, as well as the position of a ball, using visual input. I looked into computer vision frameworks and eventually chose OpenCV. I expected there to be some prebuilt functionality for something as mundane as detecting an arrow but was disappointed.

In this article I want to present the approach I took to detect the features I needed; and how this was applied in a robotics project. This article is positioned as an introduction into image detection and does not assume familiarity with the subject. Concepts and terminology are introduced as needed; it serves as a quick overview over image detection and OpenCV in particular.

\section{Problem}

\section{Overview over Computer Vision}

\section{Overview over OpenCV}

\section{Solution}

\section{Application}

\section{Conclusion}

\section{Further Reading}

\section{References}

\end{document}
